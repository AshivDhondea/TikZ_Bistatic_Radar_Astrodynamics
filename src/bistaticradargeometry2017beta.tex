% Emulates figure 6-4 in Vallado's book
% 23 February 2016
\documentclass{standalone}
\usepackage[dvipsnames]{xcolor}
\usepackage{tikz}
\usepackage{amsmath,amssymb}
\usepackage{tikz-3dplot}
\usetikzlibrary{positioning}
\usetikzlibrary{calc,fadings,decorations.pathreplacing,arrows}
\usetikzlibrary{angles,quotes}
\begin{document}


%set the plot display orientation
%syntax: \tdplotsetdisplay{\theta_d}{\phi_d}
\tdplotsetmaincoords{60}{110}

\pgfmathsetmacro{\rvec}{1.6}
\pgfmathsetmacro{\thetavec}{45}
\pgfmathsetmacro{\phivec}{60}
\pgfmathsetmacro{\sine}{\rvec*cos(\phivec)*sin(\thetavec)}
\pgfmathsetmacro{\cosine}{\rvec*cos(\phivec)*cos(\thetavec)}
\pgfmathsetmacro{\height}{\rvec*sin(\phivec)}
\pgfmathsetmacro{\betavec}{90}



\begin{tikzpicture}[scale=5,tdplot_main_coords]
\coordinate (O) at (0,0,0);
\coordinate (T) at (-1,1.1,-0.2);
\coordinate (Py) at (0,\sine,0);
\coordinate (Px) at (\cosine,0,0);
%\coordinate (Pyy) at (0.5*\cosine,\sine,0);

%\draw
%determine a coordinate (P) using (r,\theta,\phi) coordinates.  This command
%also determines (Pxy), (Pxz), and (Pyz): the xy-, xz-, and yz-projections
%of the point (P).
%syntax: \tdplotsetcoord{Coordinate name without parentheses}{r}{\theta}{\phi}
\tdplotsetcoord{P}{\rvec}{\thetavec}{\phivec}

%draw the main coordinate system axes
\draw[thick,->] (0,0,0) -- (1.2,0,0) node[anchor=north east]{$x$};
\draw[thick,->] (0,0,0) -- (0,1.6,0) node[anchor=north west]{$y$};
\draw[thick,->] (0,0,0) -- (0,0,1.1) node[anchor=south]{$z$};

\draw[very thick,shorten >=0.15cm,->] (O) -- (T) node[anchor=south]{};

%draw a vector from origin to point (P) 
\draw[very thick,color=Blue,shorten <=0.1cm,shorten >=0.2cm,->] (O) -- (P)node[anchor=south east]{};
\node[above=7pt] at (P){\textbf{RSO}};

%\node[above right] at (0.3,0.3,0.2){$\rho$};

\node[above right] at (0.15,-0.15,0){\textbf{Rx}};
\node[ right] at (T){\textbf{Tx}};
%\node[above ] at (Rxmid){$\rho_\mathbf{Rx}$};

\draw[very thick,color=OliveGreen,shorten <=0.1cm,shorten >=0.2cm,->] (T) -- (P)node[anchor=south east]{};

\draw[decoration={brace,raise=8pt},decorate](O) -- node[above left=7pt] {$\rho_\mathbf{Rx}$} (P);
\draw[decoration={brace,mirror,raise=8pt},decorate](T) -- node[above right=7pt] {$\rho_\mathbf{Tx}$} (P);
\draw[decoration={brace,mirror,raise=8pt},decorate](O) -- node[below =7pt] {$L_b$} (T);

%draw projection on xy plane, and a connecting line
\draw[dashed,thick, color=Gray] (O) -- (Pxy);
\draw[dashed,thick, color=Gray] (P) -- (Pxy);
\draw[dashed,thick,color=Gray] (Pxy) -- (Py) node[anchor= south west]{};
\draw[dashed,thick,color=Gray] (Pxy) -- (Px)node[anchor= south east]{};
%
%\node[rotate= \thetavec] at (Pyy) {\small{$\rho \cos (\theta) \cos (\psi)$}};
%$\rho \cos (\theta) \cos (\psi)$

%draw the angle \phi, and label it
%syntax: \tdplotdrawarc[coordinate frame, draw options]{center point}{r}{angle}{label options}{label}
\tdplotdrawarc[->,semithick,color=RoyalBlue]{(O)}{0.5}{0}{\phivec}%
{above}{$\psi_\mathbf{Rx}$}

\tdplotsetthetaplanecoords{\phivec}
\tdplotdrawarc[tdplot_rotated_coords,semithick,<-,color=RoyalBlue]{(O)}{0.5}{1.03*\thetavec}%
{90}{above left=6pt}{$\theta_\mathbf{Rx}$}


\shade[tdplot_screen_coords,ball color = Red] (P) circle (0.03);

\node[draw=OliveGreen,shape=circle,fill=none,inner sep=0.06 cm] (itx) at (T){};
\node[draw=Blue,shape=circle,fill=Blue,inner sep=0.06 cm] (irx) at (O){};

\pic [draw,semithick,<->,angle radius=.5cm,angle eccentricity=1.5,color=Red,"$\beta$"{anchor=east,text=Red}] 
{angle = O--P--T};



%set the rotated coordinate definition within display using a translation
%coordinate and Euler angles in the "z(\alpha)y(\beta)z(\gamma)" euler rotation convention
%syntax: \tdplotsetrotatedcoords{\alpha}{\beta}{\gamma}


\end{tikzpicture}
\end{document}

