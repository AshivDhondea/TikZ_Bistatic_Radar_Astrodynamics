\documentclass{article}
%\usepackage[table]{xcolor}
\usepackage[dvipsnames]{xcolor}
\usepackage{tikz}
%%%<
\usepackage{verbatim}
\usepackage[active,tightpage]{preview}
\PreviewEnvironment{tikzpicture}
\setlength\PreviewBorder{10pt}%
%%%>


\usetikzlibrary{arrows,shapes,positioning,shadows,trees}

\tikzset{
  basic/.style  = {draw, text width=3cm, drop shadow, font=\sffamily, rectangle},
  root/.style   = {basic, rounded corners=2pt, thin, align=center,
                   fill=SeaGreen!50},
  level 2/.style = {basic, rounded corners=6pt, thin,align=center, fill=Turquoise!50,
                   text width=12.5em},
  level 3/.style = {basic, thin, align=left, fill=Melon!50, text width=11em}
}

\begin{document}
\begin{tikzpicture}[
  level 1/.style={sibling distance=60mm},
  edge from parent/.style={->,draw},
  >=latex]

% root of the the initial tree, level 1
\node[root] {Orbit Propagation}
% The first level, as children of the initial tree
  child {node[level 2] (c1) {Analytical}}
  child {node[level 2] (c2) {Semi-analytical}}
  child {node[level 2] (c3) {Numerical}};

% The second level, relatively positioned nodes
\begin{scope}[every node/.style={level 3}]
\node [below of = c1, xshift=34pt] (c11) {General Perturbations};
\node [below of = c11] (c12) {Closed-form solutions};
%\node [below of = c12] (c13) {Simplified dynamics};
%\node [below of = c13] (c14) {Suitable for short-term OP only};
\node [below of = c12] (c13) {e.g. SGP4/SDP4};

\node [below of = c2, xshift=34pt] (c21) {Hybrid method};
%\node [below of = c21] (c22) {Use both closed-form expression \& numerical techniques};
%\node [below of = c22] (c23) {Full dynamics};
%\node [below of = c23] (c24) {Preferred option for OP over long durations};
\node [below of = c21] (c22) {e.g. DSST};

\node [below of = c3, xshift=38pt] (c31) {Special Perturbations};
\node [below of = c31] (c32) {Numerical integration of EOM};
%\node [below of = c32] (c33) {Simplified or full EOM};
%\node [below of = c33] (c34) {Time span for OP depends on chosen EOM};
\node [below of = c32] (c33) {e.g. Cowell's formulation};
\node [below of = c33] (c34) {Ideal for integration in OD routines};
\end{scope}

% lines from each level 1 node to every one of its "children"
\foreach \value in {1,2,3}
  \draw[->] (c1.195) |- (c1\value.west);

\foreach \value in {1,2}
  \draw[->] (c2.195) |- (c2\value.west);

\foreach \value in {1,2,3,4}
  \draw[->] (c3.195) |- (c3\value.west);
\end{tikzpicture}
\end{document}
