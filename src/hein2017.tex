%% Copyright 2009 Jeffrey D. Hein
%
% This work may be distributed and/or modified under the
% conditions of the LaTeX Project Public License, either version 1.3
% of this license or (at your option) any later version.
% The latest version of this license is in
%   http://www.latex-project.org/lppl.txt
% and version 1.3 or later is part of all distributions of LaTeX
% version 2005/12/01 or later.
%
% This work has the LPPL maintenance status `maintained'.
% 
% The Current Maintainer of this work is Jeffrey D. Hein.
%
% This work consists of the files 3dplot.sty and 3dplot.tex

%Description
%-----------
%3dplot.tex - an example file demonstrating the use of the 3dplot.sty package.

%Created 2009-11-07 by Jeff Hein.  Last updated: 2009-11-09
%----------------------------------------------------------

%Update Notes
%------------

%2009-11-07: Created file along with 3dplot.sty package


\documentclass{standalone}
\usepackage{amssymb,amsmath}
\usepackage[dvipsnames]{xcolor}

\usepackage{tikz}   %TikZ is required for this to work.  Make sure this exists before the next line

\usepackage{tikz-3dplot} %requires 3dplot.sty to be in same directory, or in your LaTeX installation

%\usepackage[active,tightpage]{preview}  %generates a tightly fitting border around the work
%\PreviewEnvironment{tikzpicture}
%\setlength\PreviewBorder{2mm}

\begin{document}

%Angle Definitions
%-----------------

%set the plot display orientation
%synatax: \tdplotsetdisplay{\theta_d}{\phi_d}
\tdplotsetmaincoords{60}{110}

%define polar coordinates for some vector
%TODO: look into using 3d spherical coordinate system
\pgfmathsetmacro{\rvec}{.36}
\pgfmathsetmacro{\rve}{1.1}
\pgfmathsetmacro{\thetavec}{30}
\pgfmathsetmacro{\phivec}{60}

%start tikz picture, and use the tdplot_main_coords style to implement the display 
%coordinate transformation provided by 3dplot
\begin{tikzpicture}[scale=5,tdplot_main_coords]

% Teken eerst de bol
\shade[tdplot_screen_coords,ball color = Cerulean] (0,0) circle (\rvec);

%set up some coordinates 
%-----------------------
\coordinate (O) at (0,0,0);

%determine a coordinate (P) using (r,\theta,\phi) coordinates.  This command
%also determines (Pxy), (Pxz), and (Pyz): the xy-, xz-, and yz-projections
%of the point (P).
%syntax: \tdplotsetcoord{Coordinate name without parentheses}{r}{\theta}{\phi}
\tdplotsetcoord{P}{1.05*\rve}{\thetavec}{\phivec}

\tdplotsetcoord{Q}{0.7*\rve}{0.9*\thetavec}{\phivec}
\tdplotsetcoord{o}{0.1*\rvec}{0.1*\thetavec}{0.1*\phivec}

%draw figure contents
%--------------------

%draw the main coordinate system axes
%\draw[thick,->] (0,0,0) -- (1,0,0) node[anchor=north east]{$x''$};
%\draw[thick,->] (0,0,0) -- (0,1,0) node[anchor=north west]{$y''$};
%\draw[thick,->] (0,0,-1) -- (0,0,1) node[anchor=south]{$z''$};


\draw[thick,->] (0,0,0) -- (1.4,0,0) node[anchor=north east]{$x_\text{ECEF}$};
\draw[thick,->] (0,0,0) -- (0,1.4,0) node[anchor=north west]{$y_\text{ECEF}$};
\draw[thick,->] (0,0,0) -- (0,0,1.4) node[anchor=south]{$z_\text{ECEF}$};

%draw a vector from origin to point (P) 
%\draw[-stealth,color=black] (O) -- (P) node[midway,above] {$r$};

%draw projection on xy plane, and a connecting line
%\draw[dashed, color=red] (O) -- (Pxy);
%\draw[dashed, color=red] (P) -- (Pxy);

%draw projection on xy plane, and a connecting line
%\draw[dashed, color=black] (O) -- (Pxy);
%\draw[dashed, color=black] (P) -- (Pxy);

%draw the angle \phi, and label it
%syntax: \tdplotdrawarc[coordinate frame, draw options]{center point}{r}{angle}{label options}{label}
%\tdplotdrawarc{(O)}{0.2}{0}{\phivec}{anchor=north}{$\alpha$}

\tdplotsetthetaplanecoords{1.1*\phivec}
\draw[dashed,thick,color=RoyalBlue,fill=none,tdplot_rotated_coords] (\rve,0,0) arc (0:360:\rve);


\tdplotsetrotatedcoords{\phivec}{\thetavec}{0}

%translate the rotated coordinate system
%syntax: \tdplotsetrotatedcoordsorigin{point}
%\tdplotsetrotatedcoordsorigin{(P)}

\draw (P) node[anchor=south west]{\textbf{RSO}};
%\node[draw=Purple,shape=circle,fill=Purple] (0.05*\rve) at (P){};

\shade[tdplot_screen_coords,ball color = Purple] (P) circle (0.1*\rvec);


%\node[draw=NavyBlue,shape=circle,fill=Emerald, inner sep=cos(\thetavec)*\rvec cm] (\rvec) at (0,0,0){};  % circle




%draw a vector from origin to point (P) 
\draw[-stealth,very thick,color=Red] (O) -- (P);%{$\mathbf{r}$};
\draw[color=red] (Q) node {$\vec{\mathbf{r}}$};
\draw (o) node[anchor=north east] {$0$};

%\draw[-stealth,very thick,color=Green,rotate around=90] (O) -- (P);

%set the rotated coordinate system so the x'-y' plane lies within the
%"theta plane" of the main coordinate system
%syntax: \tdplotsetthetaplanecoords{\phi}
%\tdplotsetthetaplanecoords{\phivec}

%draw theta arc and label, using rotated coordinate system
%\tdplotdrawarc[tdplot_rotated_coords]{(0,0,0)}{0.5}{\thetavec}{90}{anchor=south west}{$\beta$}

%de slechte
%\tdplotdrawarc[tdplot_rotated_coords]{(0,0,0)}{\rvec}{-180}{180}{anchor=south west}{$\gamma$}



\end{tikzpicture}

\end{document}
