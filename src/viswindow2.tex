\documentclass[landscape,12pt]{standalone}
\usepackage[utf8x]{inputenc} % utf8 encoding
\usepackage[T1]{fontenc} % use T1 fonts
\usepackage{bm} % bold math
\usepackage[dvipsnames]{xcolor}   

\usepackage{tikz}
\usetikzlibrary{shapes,arrows,shadows}
\usepackage{amsmath,bm,times,amssymb}
%%%<
\usepackage{verbatim}
\usepackage[active,tightpage]{preview}
\PreviewEnvironment{tikzpicture}
\setlength\PreviewBorder{5pt}%
%%%>
\usetikzlibrary{decorations.pathmorphing} % for snake lines
\usetikzlibrary{matrix} % for block alignment
\usetikzlibrary{calc} % for manimulation of coordinates

\usetikzlibrary{shapes.geometric}

% TikZ styles for drawing
\tikzstyle{block} = [draw,rectangle,thick,minimum height=4.em,minimum width=7.5em,  text width=10em,text centered,drop shadow, fill =Melon!20]
\tikzstyle{sum} = [draw,circle,inner sep=0mm,minimum size=2mm]
\tikzstyle{connector} = [->,thick]
\tikzstyle{line} = [thick]
\tikzstyle{branch} = [circle,inner sep=0pt,minimum size=1mm,fill=black,draw=black]
\tikzstyle{guide} = []
\tikzstyle{snakeline} = [connector, decorate, decoration={pre length=0.2cm,
                         post length=0.2cm, snake, amplitude=.4mm,
                         segment length=2mm},thick, OliveGreen, ->]

\tikzstyle{io} = [draw,trapezium, trapezium left angle=70, trapezium right angle=110,thick ,minimum height=1.9em,text width=12em,text centered,drop shadow,fill=Cerulean!20]

\renewcommand{\vec}[1]{\ensuremath{\boldsymbol{#1}}} % bold vectors
\def \myneq {\skew{-2}\not =} % \neq alone skews the dash

\begin{document}

  \begin{tikzpicture}[scale=1, auto, >=stealth']
    \small
    % node placement with matrix library: 5x4 array
    \matrix[ampersand replacement=\&, row sep=0.2cm, column sep=0.5cm] {
      %
      \node[io] (F1) {\textbf{Input TLE}}; \&
      \&
      \node[block] (f1) {\textbf{Orbit Propagation}}; \& 
      \&
      \node[block] (f2) {\textbf{Passage Identification}};\&
      \&
      \node[io] (f3) {\textbf{Visibility window interval}};\&
      \\
      \node[guide] (i1) {}; \& \& \& \& \\
    };

    % now link the nodes
   % \draw [line] (F1) -- (u1);
    \draw [connector] (F1) -- node {} (f1);

    \draw [connector] (f1) -- node {} (f2);
    \draw [connector] (f2) -- node {} (f3);
    



  \end{tikzpicture}

\end{document}
